% Math Packages
\usepackage{amsmath}
\usepackage{amssymb}
\usepackage{nicematrix}
\usepackage{nicefrac}

\usepackage{nccmath} % must be loaded before mathtools
\usepackage{mathtools}
\usepackage{steinmetz} % \phase{} for phasors

\usepackage{faktor}
\usepackage{cancel}


% Math spacing
\newlength{\spreadsize}
\setlength\spreadsize{0.5\baselineskip}

\newlength{\spreadsizeSmall}
\setlength\spreadsizeSmall{0.5\spreadsize}

% NiceMatrix Options
\NiceMatrixOptions{
    cell-space-top-limit = 5pt,
    cell-space-bottom-limit = 5pt
}

\NewDocumentEnvironment{params}{}{
    \begin{itemize}[noitemsep, topsep=0pt]
        }{%
    \end{itemize}
}
\NewDocumentCommand{\param}{mm}{
    \item[$#1$]: #2
}

% SI
\usepackage{siunitx}
\sisetup{
    per-mode=symbol, % Use a slash for "per" as in km/s, rather than km*s^-1
    inter-unit-product = \ensuremath{{}\cdot{}}, % Use a center dot for products of units
    detect-weight=true, detect-family=true, % Use the surrounding font
    output-complex-root = \textrm{j},
    complex-root-position = before-number
}

% Differentials
\usepackage{diffcoeff}
\diffdef{}
{
    op-symbol = \mathrm{d},
    *-op-left = true
}

% Use display mode for fractions (double-display fraction)
\NewDocumentCommand{\ddfrac}{mm}{
    \frac{\displaystyle #1\vphantom{\big|}}{\displaystyle #2\vphantom{\Big|}}
}

% Vectors: Use bar instead of arrow
\renewcommand{\vec}[1]{\bar{#1}}

% Curly epsilon and phi
\let\epsilonold\epsilon
\renewcommand{\epsilon}{\varepsilon}

% \let\phiold\phi
% \renewcommand{\phi}{\varphi}

% Equal sign (esp for tikz labels)
\newcommand*{\equal}{=}

% Brackets 
\NewDocumentCommand{\paren}{m}{\left(#1\right)}
\NewDocumentCommand{\brak}{m}{\left[#1\right]}
\NewDocumentCommand{\abs}{m}{\left|#1\right|}
\NewDocumentCommand{\abrak}{m}{\left\langle#1\right\rangle}
\NewDocumentCommand{\cbrak}{m}{\left\{#1\right\}}

\NewDocumentCommand{\evalat}{m O{} m}{
    \left. #3\right\vert_{#1}^{#2}
}

% Laplace and Fourier transform
\NewDocumentCommand{\laplace}{m}{
    \mathcal{L}\cbrak{#1}
}
\NewDocumentCommand{\laplaceinv}{m}{
    \mathcal{L}^{-1}\cbrak{#1}
}




% Notes
\newcommand*{\note}[1]{\text{\,,}\quad\text{#1}}
\newcommand*{\noteB}[1]{\text{\,,}\quad\textbf{#1}}
\newcommand*{\noteS}[1]{\text{\,,}\hspace{0.5em}\text{#1}}
\newcommand*{\noteSB}[1]{\text{\,,}\hspace{0.5em}\textbf{#1}}

% long note (intertext), adds extra vertical space unless starred
\NewDocumentCommand{\noteL}{sm}{
    \intertext{\quad #2}
    \IfBooleanTF{#1}{}{\intertext{}}
}

% Abbreviations
\NewDocumentCommand{\sinT}{}{\sin\theta}
\NewDocumentCommand{\cosT}{}{\cos\theta}


% Number (from siunitx) in parentheses
\NewDocumentCommand{\nump}{m}{\paren{\num{#1}}}
\NewDocumentCommand{\numb}{m}{\brak{\num{#1}}}

\usepackage{etoolbox,siunitx}
\robustify\bfseries

